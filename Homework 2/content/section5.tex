\section{Bài 5}

\subsection{Đề bài}

Cho biết các trạng thái sau là tách được hay vướng, nếu tách được thì biểu diễn trên mặt cầu Bloch
\begin{itemize}
    \item[(a)] $\frac{1}{\sqrt{2}}\left(\ket{01} + \ket{10}\right)$.
    \item[(b)] $\frac{1}{\sqrt{2}}\left(\ket{10} + i\ket{11}\right)$.
    \item[(c)] $\frac{1}{4}\left(3\ket{00} - \sqrt{3}\ket{01} + \sqrt{3}\ket{10} - \ket{11}\right)$.
    \item[(d)] $\frac{1}{\sqrt{3}}\ket{0}\ket{+} + \sqrt{\frac{2}{3}}\ket{1}\ket{-}$.   
\end{itemize}

\subsection{Lời giải}

\subsubsection{Phần a}

Giả sử $\ket{\phi_1}$ tách được, tức $\exists \ket{\psi_1}, \ket{\psi_2}$ sao cho $\ket{\phi_1} = \ket{\psi_1} \otimes \ket{\psi_2}$. Ta có:
\begin{align*}
    \ket{\phi_1} &= \begin{bmatrix}
        a \\ b
    \end{bmatrix}
    \otimes
    \begin{bmatrix}
        c \\ d
    \end{bmatrix} \\
    \Leftrightarrow 
    \begin{bmatrix}
        0 \\ \frac{1}{\sqrt{2}} \\ \frac{1}{\sqrt{2}} \\ 0
    \end{bmatrix}
    &= 
    \begin{bmatrix}
        ac \\ ad \\ bc \\ bd
    \end{bmatrix} \\
    \Rightarrow &
    \begin{cases}
        ac = bd = 0 \\
        ad = bc = \frac{1}{\sqrt{2}}
    \end{cases}
\end{align*}

$\Rightarrow$ vô lí $\Rightarrow \ket{\phi_1}$ là trạng thái vướng. 

\subsubsection{Phần b}

Ta có:
\begin{align*}
    \ket{\phi_2} &= \frac{1}{\sqrt{2}} \left( \ket{10} + i\ket{11} \right) \\
    &= \ket{1} \otimes \left( \frac{1}{\sqrt{2}} \ket{0} + \frac{i}{\sqrt{2}} \ket{1} \right) \\
    \Rightarrow \ket{\phi_2} &= \ket{1} \otimes \ket{i}
\end{align*}

$\Rightarrow \ket{\phi_2}$ tách được, ta có dạng Bloch của $\ket{\phi_2}$:
\[
\begin{cases}
    \ket{1} = \cos\left( \frac{\pi}{2} \right) \ket{0} + e^{i0} \sin\left( \frac{\pi}{2} \right)\ket{1} \\
    \ket{i} = \cos\left( \frac{\pi}{4} \right) \ket{0} + e^{i\frac{\pi}{2}} \sin\left( \frac{\pi}{4} \right)\ket{1}
\end{cases} 
\Rightarrow
\begin{cases}
    \theta_1 = \pi, \quad \phi_1 = 0 \\
    \theta_2 = \frac{\pi}{2}, \quad \phi_1 = \frac{\pi}{2} \\
\end{cases}
\]

\textbf{Mặt cầu Bloch:}


\begin{figure}[H]
  \centering
  \begin{subfigure}{0.48\textwidth}
    \centering
    \begin{blochsphere}[radius=4cm,tilt=30,rotation=-110,opacity=0.08]
        % Vẽ mặt cầu
        \drawBallGrid[style={opacity=0.2}]{30}{30};

        % Vẽ các trục tọa độ
        %% Trục Z
        \drawAxis[style={gray, dashed}]{0}{0};
        %% Trục Y
        \drawAxis[style={gray, dashed}]{90}{90};
        %% Trục X
        \drawAxis[style={gray, dashed}]{90}{0}; 
        
        \node at (0, 4.5) {$\ket{0}$};
        \node at (0, -4.5) {$\ket{1}$};
        \node at (4, -1) {$y$};
        \node at (-1.5, -2) {$x$};

        % Vẽ các trạng thái lượng tử (phi = -phi theo convention của gói blochsphere)
        \drawStatePolar[statecolor=blue, statewidth=1.5pt]{phi2_1}{180}{0};
    \end{blochsphere}
    \caption{Qubit 0}\label{fig:phi2_1}
    \end{subfigure}
    \hfill
    \begin{subfigure}{0.48\textwidth}
    \centering
    \begin{blochsphere}[radius=4cm,tilt=30,rotation=-110,opacity=0.08]
        % Vẽ mặt cầu
        \drawBallGrid[style={opacity=0.2}]{30}{30};

        % Vẽ các trục tọa độ
        %% Trục Z
        \drawAxis[style={gray, dashed}]{0}{0};
        %% Trục Y
        \drawAxis[style={gray, dashed}]{90}{90};
        %% Trục X
        \drawAxis[style={gray, dashed}]{90}{0}; 
        
        \node at (0, 4.5) {$\ket{0}$};
        \node at (0, -4.5) {$\ket{1}$};
        \node at (4, -1) {$y$};
        \node at (-1.5, -2) {$x$};

        % Vẽ các trạng thái lượng tử (phi = -phi theo convention của gói blochsphere)
        \drawStatePolar[statecolor=red, statewidth=1.5pt]{phi2_2}{90}{-90};
    \end{blochsphere}
    \caption{Qubit 1}\label{fig:phi2_2}
    \end{subfigure}
\end{figure}

\subsubsection{Phần c}
Ta có:
\begin{align*}
    \ket{\phi_3} &= \frac{1}{4} \left( 3\ket{00} - \sqrt{3}\ket{01} + \sqrt{3}\ket{10} - \ket{11} \right) \\
    &= \frac{1}{4} \left[ \sqrt{3} \ket{0} \otimes \left( \sqrt{3} \ket{0} - \ket{1} \right) + \ket{1} \otimes \left( \sqrt{3} \ket{0} - \ket{1} \right) \right] \\
    &= \frac{1}{4} \left( \sqrt{3}\ket{0} + \ket{1} \right) \otimes \left( \sqrt{3} \ket{0} - \ket{1} \right) \\
    \Rightarrow \ket{\phi_3} &= \left( \frac{\sqrt{3}}{2} \ket{0} + \frac{1}{2} \ket{1} \right) \otimes \left( \frac{\sqrt{3}}{2} \ket{0} - \frac{1}{2} \ket{1} \right)
\end{align*}

$\Rightarrow \ket{\phi_3}$ tách được, ta có dạng Bloch của $\ket{\phi_3}$:
\[
\begin{cases}
    \ket{\psi_1} = \frac{\sqrt{3}}{2} \ket{0} + \frac{1}{2} \ket{1} = \cos\left( \frac{\pi}{6} \right) \ket{0} + e^{i0} \sin\left( \frac{\pi}{6} \right)\ket{1} \\
    \ket{\psi_2} = \frac{\sqrt{3}}{2} \ket{0} - \frac{1}{2} \ket{1} = \cos\left( \frac{\pi}{6} \right) \ket{0} + e^{i\pi} \sin\left( \frac{\pi}{6} \right)\ket{1}
\end{cases}
\Rightarrow
\begin{cases}
    \theta_1 = \frac{\pi}{3}, \quad \phi_1 = 0 \\
    \theta_2 = \frac{\pi}{3}, \quad \phi_2 = \pi \\
\end{cases}
\]

\textbf{Mặt cầu Bloch:}


\begin{figure}[H]
  \centering
  \begin{subfigure}{0.48\textwidth}
    \centering
    \begin{blochsphere}[radius=4cm,tilt=30,rotation=-110,opacity=0.08]
        % Vẽ mặt cầu
        \drawBallGrid[style={opacity=0.2}]{30}{30};

        % Vẽ các trục tọa độ
        %% Trục Z
        \drawAxis[style={gray, dashed}]{0}{0};
        %% Trục Y
        \drawAxis[style={gray, dashed}]{90}{90};
        %% Trục X
        \drawAxis[style={gray, dashed}]{90}{0}; 
        
        \node at (0, 4.5) {$\ket{0}$};
        \node at (0, -4.5) {$\ket{1}$};
        \node at (4, -1) {$y$};
        \node at (-1.5, -2) {$x$};

        % Vẽ các trạng thái lượng tử (phi = -phi theo convention của gói blochsphere)
        \drawStatePolar[statecolor=blue, statewidth=1.5pt]{phi3_1}{60}{0};
    \end{blochsphere}
    \caption{Qubit 0}
    \label{fig:phi3_1}
    \end{subfigure}
    \hfill
    \begin{subfigure}{0.48\textwidth}
    \centering
    \begin{blochsphere}[radius=4cm,tilt=30,rotation=-110,opacity=0.08]
        % Vẽ mặt cầu
        \drawBallGrid[style={opacity=0.2}]{30}{30};

        % Vẽ các trục tọa độ
        %% Trục Z
        \drawAxis[style={gray, dashed}]{0}{0};
        %% Trục Y
        \drawAxis[style={gray, dashed}]{90}{90};
        %% Trục X
        \drawAxis[style={gray, dashed}]{90}{0}; 
        
        \node at (0, 4.5) {$\ket{0}$};
        \node at (0, -4.5) {$\ket{1}$};
        \node at (4, -1) {$y$};
        \node at (-1.5, -2) {$x$};

        % Vẽ các trạng thái lượng tử (phi = -phi theo convention của gói blochsphere)
        \drawStatePolar[statecolor=red, statewidth=1.5pt]{phi3_2}{60}{-180};
    \end{blochsphere}
    \caption{Qubit 1}
    \label{fig:phi3_2}
    \end{subfigure}
\end{figure}

\subsubsection{Phần d}

Ta có:
\begin{align*}
    \ket{\phi_4} &= \frac{1}{\sqrt{3}}\ket{0}\ket{+} + \sqrt{\frac{2}{3}}\ket{1}\ket{-} \\
    &= \frac{1}{\sqrt{3}} \ket{0} \otimes \left( \frac{1}{\sqrt{2}} \ket{0} + \frac{1}{\sqrt{2}} \ket{1} \right) + \sqrt{\frac{2}{3}} \ket{1} \otimes \left( \frac{1}{\sqrt{2}} \ket{0} - \frac{1}{\sqrt{2}} \ket{1} \right) \\
    &= \frac{1}{\sqrt{6}} \ket{00} + \frac{1}{\sqrt{6}} \ket{01} + \frac{1}{\sqrt{3}} \ket{10} -\frac{1}{\sqrt{3}} \ket{11} \\
\end{align*}

Giả sử $\ket{\phi_4}$ tách được, tức $\exists \ket{\psi_1}, \ket{\psi_2}$ sao cho $\ket{\phi_4} = \ket{\psi_1} \otimes \ket{\psi_2}$. Ta có:
\begin{align*}
    \ket{\phi_4} &= \begin{bmatrix}
        a \\ b
    \end{bmatrix}
    \otimes
    \begin{bmatrix}
        c \\ d
    \end{bmatrix} \\
    \Leftrightarrow 
    \begin{bmatrix}
        \frac{1}{\sqrt{6}} \\ \frac{1}{\sqrt{6}} \\ \frac{1}{\sqrt{3}} \\ -\frac{1}{\sqrt{3}}
    \end{bmatrix}
    &= 
    \begin{bmatrix}
        ac \\ ad \\ bc \\ bd
    \end{bmatrix} \\
    \Rightarrow &
    \begin{cases}
        ac = ad = \frac{1}{\sqrt{6}} \\
        bc = -bd = \frac{1}{\sqrt{3}}
    \end{cases} \\
    \Rightarrow &
    \begin{cases}
        \frac{ac}{ad} = \frac{c}{d} = 1 \\
        \frac{bc}{bd} = \frac{c}{d} = -1
    \end{cases}
\end{align*}

$\Rightarrow$ vô lí $\Rightarrow \ket{\phi_4}$ là trạng thái vướng.