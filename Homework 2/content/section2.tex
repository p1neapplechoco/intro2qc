\section{Bài 2}

\subsection{Đề bài}

Viết dạng Bloch và mô tả trên mặt cầu Bloch các trạng thái lượng tử ở Câu 1.

\subsection{Lời giải}

Ta có:

\[
\begin{cases}
    \ket{\psi_1} = \frac{\sqrt{3}}{2}\ket{0} + \frac{1}{2}\ket{1} \\
    \ket{\psi_2} = \frac{1}{\sqrt{2}}\ket{0} + \frac{1}{\sqrt{2}} e^{i\frac{\pi}{6}}\ket{1} \\
    \ket{\psi_3} = \frac{2}{3}\ket{0} + \frac{1 - 2i}{3}\ket{1}
\end{cases}
\Leftrightarrow
\begin{cases}
    \ket{\psi_1} = \frac{\sqrt{3}}{2}\ket{0} + \frac{1}{2} e^{0i} \ket{1} \\
    \ket{\psi_2} = \frac{1}{\sqrt{2}}\ket{0} + \frac{1}{\sqrt{2}} e^{i\frac{\pi}{6}}\ket{1} \\
    \ket{\psi_3} = \frac{2}{3}\ket{0} + \frac{\sqrt{5}}{3} e^{i \arctan(-2)} \ket{1}
\end{cases}
\]
\[
\Rightarrow
\left\{
\begin{aligned}
\theta_1 &= \frac{\pi}{3},  &\phi_1 &= 0 \\
\theta_2 &= \frac{\pi}{2},  &\phi_2 &= \frac{\pi}{6} \\
\theta_3 &= 2\arccos{\frac{2}{3}},  &\phi_3 &= \arctan(-2)
\end{aligned}
\right.
\]

$\Rightarrow$ dạng Bloch của các trạng thái lượng tử $\ket{\psi_1}, \ket{\psi_2}, \ket{\psi_3}$ là:
\[
\begin{cases}
    \ket{\psi_1} = \cos \left(\frac{\pi}{6}\right) \ket{0} + e^{i0} \sin \left(\frac{\pi}{6}\right) \ket{1} \\
    \ket{\psi_2} = \cos \left(\frac{\pi}{4}\right) \ket{0} + e^{i\frac{\pi}{6}} \sin \left(\frac{\pi}{4}\right) \ket{1} \\
    \ket{\psi_3} = \cos \left(\arccos{\frac{2}{3}}\right) \ket{0} + e^{i\arctan(-2)} \sin \left(\arccos{\frac{2}{3}}\right) \ket{1} \\
\end{cases}
\]

\textbf{Mặt cầu Bloch:}
\begin{center}
    \begin{blochsphere}[radius=4cm,tilt=30,rotation=-110,opacity=0.08]
        % Vẽ mặt cầu
        \drawBallGrid[style={opacity=0.2}]{30}{30};

        % Vẽ các trục tọa độ
        %% Trục Z
        \drawAxis[style={gray, dashed}]{0}{0};
        %% Trục Y
        \drawAxis[style={gray, dashed}]{90}{90};
        %% Trục X
        \drawAxis[style={gray, dashed}]{90}{0}; 
        
        \node at (0, 4.5) {$\ket{0}$};
        \node at (0, -4.5) {$\ket{1}$};
        \node at (4, -1) {$y$};
        \node at (-1.5, -2) {$x$};

        % Vẽ các trạng thái lượng tử (phi = -phi theo convention của gói blochsphere)
        \drawStatePolar[statecolor=red, statewidth=1.5pt]{psi1}{60}{0};
        \node at (-1, 0.5) {$\ket{\psi_1}$};

        \drawStatePolar[statecolor=blue, statewidth=1.5pt]{psi2}{90}{-30};
        \node at (1.25, -2) {$\ket{\psi_2}$};

        \drawStatePolar[statecolor=green, statewidth=1.5pt]{psi3}{96.38}{63.43};
        \node at (-4, 0) {$\ket{\psi_3}$};
    \end{blochsphere}
\end{center}