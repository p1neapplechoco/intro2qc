\section{Bài 3}

\subsection{Đề bài}

Cho $U$ là một toán tử tuyến tính trên $\mathbb{C}^2$, biết:
$$U\ket{0} = \frac{\sqrt{2} - i}{2} \ket{0} - \frac{1}{2}\ket{1},$$
$$U\ket{1} = \frac{1}{2} \ket{0} + \frac{\sqrt{2} + i}{2}\ket{1}.$$

\begin{itemize}
    \item[(a)] Chứng minh $U$ là một cổng lượng tử. 
    \item[(b)] Cho biết kết quả biến đổi $U$ trên các trạng thái $\ket{+}, \ket{-}, \ket{i}, \ket{-i}$.
    \item[(c)] Cho biết kết quả biến đổi $U$ trên các trạng thái của Câu 1.
    \item[(d)] $U$ tương ứng với phép quay quanh trục nào với góc bao nhiêu trên mặt cầu Bloch? 
\end{itemize}

\subsection{Lời giải}

\subsubsection{Phần a}
Ta có:
$$U\ket{0} = \frac{\sqrt{2} - i}{2} \ket{0} - \frac{1}{2}\ket{1}, U\ket{1} = \frac{1}{2} \ket{0} + \frac{\sqrt{2} + i}{2}\ket{1}$$
$\Rightarrow$ ma trận $U$ trong cơ sở $\ket{0}, \ket{1}$ là:
$$ U = \begin{bmatrix}
    \frac{\sqrt{2} - i}{2} & \frac{1}{2} \\
    -\frac{1}{2} & \frac{\sqrt{2} + i}{2}
\end{bmatrix}$$

Xét biểu thức:
\begin{align*}
    U^\dagger U &= 
    \begin{bmatrix} 
        \frac{\sqrt{2} + i}{2} & -\frac{1}{2} \\
        \frac{1}{2} & \frac{\sqrt{2} - i}{2}
    \end{bmatrix}
    \begin{bmatrix}
        \frac{\sqrt{2} - i}{2} & \frac{1}{2} \\
        -\frac{1}{2} & \frac{\sqrt{2} + i}{2}
    \end{bmatrix} \\
    &= \frac{1}{4} 
    \begin{bmatrix}
        \sqrt{2} + i & -1 \\
        1 & \sqrt{2} - i
    \end{bmatrix}
    \begin{bmatrix}
        \sqrt{2} - i & 1 \\
        -1 & \sqrt{2} + i
    \end{bmatrix} \\
    \Rightarrow U^\dagger U &= 
    \begin{bmatrix}
        1 & 0 \\
        0 & 1
    \end{bmatrix} = I
\end{align*}

$\Rightarrow U$ là một ma trận unita $\Rightarrow U$ là một cổng lượng tử (dpcm). 

\subsubsection{Phần b}
Ta có:
$$ \ket{+} = \frac{1}{\sqrt{2}} \left(\ket{0} + \ket{1}\right), \ket{-} = \frac{1}{\sqrt{2}} \left(\ket{0} - \ket{1}\right), $$
$$ \ket{i} = \frac{1}{\sqrt{2}} \left(\ket{0} + i\ket{1}\right), \ket{-i} = \frac{1}{\sqrt{2}} \left(\ket{0} - i\ket{1}\right), $$

\textbf{Trạng thái $\ket{+}$}:
\begin{align*}
    U\ket{+} &= \frac{1}{\sqrt{2}} \left(U\ket{0} + U\ket{1}\right) \\
    &= \frac{1}{\sqrt{2}} \left(\frac{\sqrt{2} - i}{2} \ket{0} - \frac{1}{2}\ket{1} + \frac{1}{2} \ket{0} + \frac{\sqrt{2} + i}{2}\ket{1}\right) \\
    \Rightarrow U\ket{+} &= \frac{\sqrt{2} + 1 - i}{2\sqrt{2}} \ket{0} + \frac{\sqrt{2} - 1 + i}{2\sqrt{2}} \ket{1}
\end{align*}

\textbf{Trạng thái $\ket{-}$}:
\begin{align*}
    U\ket{-} &= \frac{1}{\sqrt{2}} \left(U\ket{0} - U\ket{1}\right) \\
    &= \frac{1}{\sqrt{2}} \left(\frac{\sqrt{2} - i}{2} \ket{0} - \frac{1}{2}\ket{1} - \frac{1}{2} \ket{0} - \frac{\sqrt{2} + i}{2}\ket{1}\right) \\
    \Rightarrow U\ket{+} &= \frac{\sqrt{2} - 1 - i}{2\sqrt{2}}\ket{0} - \frac{\sqrt{2} + 1 +i}{2\sqrt{2}}\ket{1}
\end{align*}

\textbf{Trạng thái $\ket{i}$}:
\begin{align*}
    U\ket{i} &= \frac{1}{\sqrt{2}} \left(U\ket{0} + iU\ket{1}\right) \\
    &= \frac{1}{\sqrt{2}} \left(\frac{\sqrt{2} - i}{2}\ket{0} - \frac{1}{2}\ket{1} + \frac{i}{2}\ket{0} + \frac{i\sqrt{2} - 1}{2} \ket{1} \right) \\
    \Rightarrow U\ket{i} &= \frac{1}{2}\ket{0} + \frac{i\sqrt{2} - 2}{2}\ket{1}
\end{align*}

\textbf{Trạng thái $\ket{-i}$}:
\begin{align*}
    U\ket{-i} &= \frac{1}{\sqrt{2}} \left(U\ket{0}- iU\ket{1}\right) \\
    &= \frac{1}{\sqrt{2}} \left(\frac{\sqrt{2} - i}{2}\ket{0} - \frac{1}{2}\ket{1} - \frac{i}{2}\ket{0} - \frac{i\sqrt{2} - 1}{2}\ket{1}\right) \\
    \Rightarrow U\ket{-i} &= \frac{\sqrt{2} - 2i}{2\sqrt{2}}\ket{0} - \frac{i}{2}\ket{1}
\end{align*}

\subsubsection{Phần c}
\textbf{Trạng thái $\ket{\psi_1}$}:
\begin{align*}
    U\ket{\psi_1} &= \frac{\sqrt{3}}{2}U\ket{0} + \frac{1}{2}U\ket{1} \\
    &= \frac{\sqrt{3}}{2} \left(\frac{\sqrt{2} - i}{2} \ket{0} - \frac{1}{2} \ket{1}\right) + \frac{1}{2} \left(\frac{1}{2} \ket{0} + \frac{\sqrt{2} + i}{2} \ket{1}\right) \\
    &= \frac{\sqrt{6} - i\sqrt{3}}{4}\ket{0} - \frac{\sqrt{3}}{4} \ket{1} + \frac{1}{4} \ket{0} + \frac{\sqrt{2} + i}{4}\ket{1} \\
    \Rightarrow U\ket{\psi_1} &= \frac{\sqrt{6} + 1 - i\sqrt{3}}{4}\ket{0} + \frac{\sqrt{2} - \sqrt{3} + i}{4}\ket{1}
\end{align*}

\textbf{Trạng thái $\ket{\psi_2}$}:
\begin{align*}
    U\ket{\psi_2} &= \frac{1}{\sqrt{2}} U\ket{0} + \frac{1}{\sqrt{2}}\left(\cos{\frac{\pi}{6}} + i\sin{\frac{\pi}{6}} \right) U\ket{1} \\
    &= \frac{1}{\sqrt{2}} U\ket{0} + \left(\frac{\sqrt{6}}{4} + \frac{i\sqrt{2}}{4}\right)U \ket{1} \\
    &= \frac{1}{\sqrt{2}} \left(\frac{\sqrt{2} - i}{2} \ket{0} - \frac{1}{2} \ket{1}\right) + \left(\frac{\sqrt{6}}{4} + \frac{i\sqrt{2}}{4}\right)\left(\frac{1}{2} \ket{0} + \frac{\sqrt{2} + i}{2} \ket{1}\right) \\
    &= \frac{\sqrt{2} - i}{2\sqrt{2}} \ket{0} - \frac{1}{2\sqrt{2}} \ket{1} + \frac{\sqrt{6} + i\sqrt{2}}{8}\ket{0} + \frac{6 + 2\sqrt{3} + \left(2 + \sqrt{2}\right)i}{8} \ket{1} \\
    \Rightarrow U\ket{\psi_2} &= \frac{4 + \sqrt{6} - i\sqrt{2}}{8}\ket{0} + \frac{6 + 2\sqrt{3} + 2\sqrt{2} + i\left(2 +\sqrt{2}\right)}{8} \ket{1}
\end{align*}

\textbf{Trạng thái $\ket{\psi_3}$}:
\begin{align*}
    U\ket{\psi_3} &= \frac{2}{3} U\ket{0} + \frac{1 - 2i}{3} U\ket{1} \\
    &= \frac{2}{3} \left(\frac{\sqrt{2} - i}{2} \ket{0} - \frac{1}{2} \ket{1}\right) + \frac{1 - 2i}{3} \left(\frac{1}{2} \ket{0} + \frac{\sqrt{2} + i}{2} \ket{1}\right) \\
    &= \frac{\sqrt{2} - i}{3}\ket{0} - \frac{1}{3}\ket{1} + \frac{1 - 2i}{6} \ket{0} + \frac{2 + \sqrt{2} + i\left(1 - 2\sqrt{2}\right)}{6}\ket{1} \\
    \Rightarrow U\ket{\psi_3} &= \frac{2\sqrt{2} + 1 - 4i}{6} \ket{0} + \frac{\sqrt{2} + i\left(1 - 2\sqrt{2}\right)}{6} \ket{1} 
\end{align*}

\subsubsection{Phần d}