\section{Bài 3}

\subsection{Đề bài}
Cho $\ket{\phi} = \frac{\sqrt{3}}{2} \ket{0} + \frac{1}{2} \ket{1}, \ket{\psi} = \frac{2}{3} \ket{0} + \frac{1 - 2i}{3} \ket{1}$.

\begin{itemize}
    \item[(a)] Tính $\bra{\phi}$ và $\bra{\psi}$.
    \item[(b)] Tính $\braket{\phi}{\psi}$ và $\braket{\psi}{\phi}$.
    \item[(c)] Tính $\ket{\phi}\bra{\psi}$ và $\ket{\psi}\bra{\phi}$.
    \item[(d)] Tính $\bra{\phi}\bra{\psi}$ và $\bra{\psi}\bra{\phi}$.
    \item[(e)] Tính $\|{\phi}\|$ và $\|{\psi}\|$.
    \item[(f)] Tính góc giữa $\bra{\phi}$ và $\bra{\psi}$.
    \item[(g)] Tính $\operatorname{proj}_{\ket{\psi}}{\ket{\phi}}$ và $\operatorname{proj}_{\ket{\phi}}{\ket{\psi}}$.
    \item[(h)] Chuẩn hoá $\operatorname{proj}_{\ket{\psi}}{\ket{\phi}}$ và $\operatorname{proj}_{\ket{\phi}}{\ket{\psi}}$.
    \item[(i)] Tìm toạ độ của $\ket{\phi}$ và $\ket{\psi}$ trong các cơ sở
    \[
    B_Z = \{\ket{0}, \ket{1}\}, B_X = \{\ket{+}, \ket{-}\}, B_Y = \{\ket{i}, \ket{-i}\}. 
    \]
    \item[(j)] Cho $\ket{a} = \frac{\sqrt{3}}{2} \ket{0} + \frac{i}{2} \ket{1}, \ket{b} = \frac{i}{2} \ket{0} + \frac{\sqrt{3}}{2} \ket{1}, $ chứng minh $B = \{a, b\}$ là một cơ sở trực chuẩn của $\mathbb{C}^2$ và tìm toạ độ của $\ket{\phi}, \ket{\psi}$ theo $B$.
\end{itemize}

\subsection{Lời giải}
\subsubsection{Phần (a)}
Ta có:
\[
\bra{\phi} = \ket{\phi}^\dagger =\left( \frac{\sqrt{3}}{2} 
\begin{bmatrix}
    1 \\ 
    0 
\end{bmatrix}
+
\frac{1}{2}
\begin{bmatrix}
    0 \\
    1
\end{bmatrix}
\right)^\dagger =
\left(
    \begin{bmatrix}
        \frac{\sqrt{3}}{2} \\ 
        0 
    \end{bmatrix}
    +
    \begin{bmatrix}
        0 \\
        \frac{1}{2}
    \end{bmatrix}
\right)^\dagger =
\begin{bmatrix}
    \frac{\sqrt{3}}{2} \\
    \frac{1}{2}
\end{bmatrix}^\dagger =
\begin{bmatrix}
    \dfrac{\sqrt{3}}{2} & \dfrac{1}{2}
\end{bmatrix}
\]
\[
\bra{\psi} = \ket{\psi}^\dagger =\left( \frac{2}{3} 
\begin{bmatrix}
    1 \\ 
    0 
\end{bmatrix}
+
\frac{1 - 2i}{3}
\begin{bmatrix}
    0 \\
    1
\end{bmatrix}
\right)^\dagger =
\left(
    \begin{bmatrix}
        \frac{2}{3} \\ 
        0 
    \end{bmatrix}
    +
    \begin{bmatrix}
        0 \\
        \frac{1 - 2i}{3}
    \end{bmatrix}
\right)^\dagger =
\begin{bmatrix}
    \frac{2}{3} \\
    \frac{1 - 2i}{3}
\end{bmatrix}^\dagger =
\begin{bmatrix}
    \dfrac{2}{3} & \dfrac{1 + 2i}{3}
\end{bmatrix}
\]

\subsubsection{Phần (b)}
Ta có:
\[
\braket{\phi}{\psi} = \bra{\phi} \ket{\psi} = 
\begin{bmatrix}
    \dfrac{\sqrt{3}}{2} & \dfrac{1}{2}
\end{bmatrix}
\begin{bmatrix}
    \frac{2}{3} \\
    \frac{1 - 2i}{3}
\end{bmatrix}
=
\frac{\sqrt{3}}{2} \times \frac{2}{3} + \frac{1}{2} \times \frac{1 - 2i}{3}
= \frac{1 + 2\sqrt{3}}{6} - \frac{1}{3}i
\]
\[
\braket{\psi}{\phi} = \bra{\psi} \ket{\phi} = 
\begin{bmatrix}
    \dfrac{2}{3} & \dfrac{1 + 2i}{3}
\end{bmatrix}
\begin{bmatrix}
    \frac{\sqrt{3}}{2} \\
    \frac{1}{2}
\end{bmatrix}
=
\frac{2}{3} \times \frac{\sqrt{3}}{2} + \frac{1 + 2i}{3} \times \frac{1}{2}
= \frac{1 + 2\sqrt{3}}{6} + \frac{1}{3}i
\]
\subsubsection{Phần (c)}
Ta có:
\[
\ket{\phi}\bra{\psi} = 
\begin{bmatrix}
    \frac{\sqrt{3}}{2} \\
    \frac{1}{2}
\end{bmatrix}
\begin{bmatrix}
    \dfrac{2}{3} & \dfrac{1 + 2i}{3}
\end{bmatrix} = 
\begin{bmatrix}
    \frac{\sqrt{3}}{2} \times \frac{2}{3} & \frac{\sqrt{3}}{2} \times \frac{1 + 2i}{3} \\
    \frac1{2} \times \frac{2}{3} & \frac{1}{2} \times \frac{1 + 2i}{3}
\end{bmatrix} =
\begin{bmatrix}
    \frac{\sqrt{3}}{3} & \frac{\sqrt{3}}{6} + \frac{\sqrt{3}}{3}i \\
    \frac{1}{3} & \frac{1}{6} + \frac{1}{3}i
\end{bmatrix}
\]
\[
\ket{\psi}\bra{\phi} = 
\left(\ket{\phi}\bra{\psi}\right)^\dagger =
\begin{bmatrix}
    \frac{\sqrt{3}}{3} & \frac{\sqrt{3}}{6} + \frac{\sqrt{3}}{3}i \\
    \frac{1}{3} & \frac{1}{6} + \frac{1}{3}i
\end{bmatrix}^\dagger =
\begin{bmatrix}
    \frac{\sqrt{3}}{3} & \frac{1}{3} \\
    \frac{\sqrt{3}}{6} - \frac{\sqrt{3}}{3}i & \frac{1}{6} - \frac{1}{3}i
\end{bmatrix}
\]

\subsubsection{Phần (d)}
Ta có:
\[
\bra{\phi}\bra{\psi} = 
\begin{bmatrix}
    \frac{\sqrt{3}}{2} &
    \frac{1}{2}
\end{bmatrix}
\otimes
\begin{bmatrix}
    \frac{2}{3} &
    \frac{1 + 2i}{3}
\end{bmatrix} = 
\begin{bmatrix}
    \frac{\sqrt{3}}{2} \times \frac{2}{3} &
    \frac{\sqrt{3}}{2} \times \frac{1 + 2i}{3} &
    \frac{1}{2} \times \frac{2}{3} &
    \frac{1}{2} \times \frac{1 + 2i}{3} 
\end{bmatrix} =
\begin{bmatrix}
    \frac{\sqrt{3}}{3} &
    \frac{\sqrt{3}}{6} + \frac{\sqrt{3}}{3}i &
    \frac{1}{3} &
    \frac{1}{6} + \frac{1}{3}i
\end{bmatrix} 
\]
\[
\bra{\psi}\bra{\phi} = 
\begin{bmatrix}
    \frac{2}{3} &
    \frac{1 + 2i}{3}
\end{bmatrix}
\otimes 
\begin{bmatrix}
    \frac{\sqrt{3}}{2} &
    \frac{1}{2}
\end{bmatrix} =
\begin{bmatrix}
    \frac{2}{3} \times \frac{\sqrt{3}}{2} &
    \frac{2}{3} \times \frac{1}{2} &
    \frac{1 + 2i}{3} \times \frac{\sqrt{3}}{2} &
    \frac{1 + 2i}{3} \times \frac{1}{2} 
\end{bmatrix} =
\begin{bmatrix}
    \frac{\sqrt{3}}{3} &
    \frac{1}{3} &
    \frac{\sqrt{3}}{6} + \frac{\sqrt{3}}{3}i &
    \frac{1}{6} + \frac{1}{3}i
\end{bmatrix} 
\]

\subsubsection{Phần (e)}
Ta có:
\[
\|\phi\| = \sqrt{\left|\frac{\sqrt{3}}{2}\right|^2 + \left|\frac{1}{2}\right|^2} = \sqrt{\frac{3}{4} +\frac{1}{4}} = \sqrt{1} = 1
\]
\[
\|\psi\| = \sqrt{\left|\frac{2}{3}\right|^2 + \left|\frac{1 + 2i}{3}\right|^2} = \sqrt{\frac{4}{9} +\frac{5}{9}} = \sqrt{1} = 1
\]

\subsubsection{Phần (f)}
Gọi $\theta$ là góc giữa $\ket{\phi}$ và $\ket{\psi}$, ta có:
\[
\cos \theta = \frac{\left| \langle \phi, \psi \rangle \right|}{\left| \phi \right| \left|\psi \right|} = \frac{\left| \braket{\phi}{\psi} \right|}{\| \phi \| \|\psi \|}
\]
\[
\Rightarrow \theta = \cos^{-1} \frac{\left| \braket{\phi}{\psi} \right|}{\| \phi \| \|\psi \|} = \cos^{-1} \frac{\left| \frac{1 + 2\sqrt{3}}{6} - \frac{1}{3}i\right|}{1 \times 1} = \cos^{-1} \sqrt{\left(\frac{1 + 2\sqrt{3}}{6}\right)^2 + \left(\frac{1}{3}\right)^2} = \cos^{-1} \sqrt{\frac{17 + 4\sqrt{3}}{36}}
\]

\subsubsection{Phần (g)}
Ta có:
\[
\operatorname{proj}_{\ket{\psi}}{\ket{\phi}} = \frac{\braket{\psi}{\phi}}{\|\ket{\psi}\|^2}\ket{\psi} = \frac{\frac{1 + 2\sqrt{3}}{6} + \frac{1}{3}i}{1} 
\begin{bmatrix}
    \frac{2}{3} \\
    \frac{1 - 2i}{3}
\end{bmatrix} =
\begin{bmatrix}
    \frac{1 + 2\sqrt{6}}{9} + \frac{2}{9}i \\
    \frac{5 + 2\sqrt{6}}{18} - \frac{2 + 2\sqrt{6}}{9} i
\end{bmatrix}
\]

\[
\operatorname{proj}_{\ket{\phi}}{\ket{\psi}} = \frac{\braket{\phi}{\psi}}{\|\ket{\phi}\|^2}\ket{\phi} = \frac{\frac{1 + 2\sqrt{3}}{6} - \frac{1}{3}i}{1} 
\begin{bmatrix}
    \frac{\sqrt{3}}{2} \\
    \frac{1}{2}
\end{bmatrix} =
\begin{bmatrix}
    \frac{\sqrt{3} + 6\sqrt{2}}{12} - \frac{\sqrt{3}}{6}i \\
    \frac{1 + 2\sqrt{6}}{12} - \frac{1}{6} i
\end{bmatrix}
\]
\subsubsection{Phần (h)}
Gọi $e_1, e_2$ lần lượt là vector chuẩn hoá của $\operatorname{proj}_{\ket{\psi}}{\ket{\phi}}, \operatorname{proj}_{\ket{\phi}}{\ket{\psi}}$, ta có:
\begin{align*}
    e_1 &= \operatorname{proj}_{\ket{\psi}}{\ket{\phi}} \frac{1}{\|\operatorname{proj}_{\ket{\psi}}{\ket{\phi}}\|} \\
    &= 
    \begin{bmatrix}
        \frac{1 + 2\sqrt{6}}{9} - \frac{2}{9}i \\
        \frac{5 + 2\sqrt{6}}{18} - \frac{2 + 2\sqrt{6}}{9} i
    \end{bmatrix}
    \frac{1}{\sqrt{\left|\frac{1 + 2\sqrt{6}}{9} - \frac{2}{9}i \right|^2 + \left|\frac{5 + 2\sqrt{6}}{18} - \frac{2 + 2\sqrt{6}}{9} i\right|^2}} \\
    &= 
    \begin{bmatrix}
        \frac{1 + 2\sqrt{6}}{9} - \frac{2}{9}i \\
        \frac{5 + 2\sqrt{6}}{18} - \frac{2 + 2\sqrt{6}}{9} i
    \end{bmatrix}
    \frac{1}{\left(\frac{1 + 2\sqrt{6}}{9}\right)^2 + \left(\frac{2}{9}\right)^2 + \left(\frac{5 + 2\sqrt{6}}{18}\right)^2 + \left(\frac{2 + 2\sqrt{6}}{9}\right)^2}
\end{align*}

\begin{align*}
    e_2 &= \operatorname{proj}_{\ket{\phi}}{\ket{\psi}} \frac{1}{\|\operatorname{proj}_{\ket{\phi}}{\ket{\psi}}\|} \\
    &= 
    \begin{bmatrix}
        \frac{\sqrt{3} + 6\sqrt{2}}{12} - \frac{\sqrt{3}}{6}i \\
        \frac{1 + 2\sqrt{6}}{12} - \frac{1}{6} i
    \end{bmatrix}
    \frac{1}{\sqrt{\left|\frac{\sqrt{3} + 6\sqrt{2}}{12} - \frac{\sqrt{3}}{6}i \right|^2 + \left|\frac{1 + 2\sqrt{6}}{12} - \frac{1}{6} i\right|^2}} \\
    &= 
    \begin{bmatrix}
        \frac{\sqrt{3} + 6\sqrt{2}}{12} - \frac{\sqrt{3}}{6}i \\
        \frac{1 + 2\sqrt{6}}{12} - \frac{1}{6} i
    \end{bmatrix}
    \frac{1}{\left(\frac{\sqrt{3} + 6\sqrt{2}}{12}\right)^2 + \left(\frac{\sqrt{3}}{6}\right)^2 + \left(\frac{1 + 2\sqrt{6}}{12}\right)^2 + \left(\frac{1}{6}\right)^2}
\end{align*}

\subsubsection{Phần (i)}
Ta có:
\[
\ket{\phi} = \frac{\sqrt{3}}{2} \ket{0} + \frac{1}{2} \ket{1}, \ket{\psi} = \frac{2}{3}\ket{0} + \frac{1 - 2i}{3} \ket{1}
\]
$\Rightarrow$ toạ độ của $\ket{\phi}, \ket{\psi}$ trong cơ sở trực chuẩn $B_Z$ là:
\[
\begin{cases}
    \left[\ket{\phi}\right]_{B_Z} = \left(\frac{\sqrt{3}}{2}, \frac{1}{2}\right) \\
    \left[\ket{\psi}\right]_{B_Z} = \left(\frac{2}{3}, \frac{1 - 2i}{3}\right)
\end{cases}
\]

Xét cơ sở trực chuẩn $B_X$, toạ độ của $\ket{\phi}, \ket{\psi}$ trong $B_X$ là:
\[
\begin{cases}
    \left[\ket{\phi}\right]_{B_X} = \left(\braket{+}{\phi}, \braket{-}{\phi}\right) = \left(\frac{2 + \sqrt{6}}{4}, \frac{-2 + \sqrt{6}}{4}\right) \\
    \left[\ket{\psi}\right]_{B_X} = \left(\braket{+}{\psi}, \braket{-}{\psi}\right) = \left(\frac{\sqrt{2}}{2} - \frac{\sqrt{2}}{3}i, \frac{\sqrt{2}}{6} + \frac{\sqrt{2}}{3}i\right)
\end{cases}
\]

Xét cơ sở trực chuẩn $B_Y$, toạ độ của $\ket{\phi}, \ket{\psi}$ trong $B_Y$ là:
\[
\begin{cases}
    \left[\ket{\phi}\right]_{B_Y} = \left(\braket{i}{\phi}, \braket{-i}{\phi}\right) = \left(-\frac{\sqrt{6}}{4} - \frac{\sqrt{2}}{4}i, -\frac{\sqrt{6}}{4} + \frac{\sqrt{2}}{4}i\right) \\
    \left[\ket{\psi}\right]_{B_Y} = \left(\braket{i}{\psi}, \braket{-i}{\psi}\right) = \left(\frac{2\sqrt{2}}{3} + \frac{\sqrt{2}}{6}i, -\frac{\sqrt{2}}{6}i\right)
\end{cases}
\]

\subsubsection{Phần (j)}
Gọi $\alpha_1, \alpha_2 \in \mathbb{C}$, xét biểu thức:
\begin{align*}
    &\quad \alpha_1 a + \alpha_2 b &= 0 \\
    &\Leftrightarrow \alpha_1 \left(\frac{\sqrt{3}}{2} \begin{bmatrix} 1 & 0 \end{bmatrix} +\frac{i}{2} \begin{bmatrix} 0 & 1 \end{bmatrix} \right) + \alpha_2 \left(\frac{i}{2} \begin{bmatrix} 1 & 0 \end{bmatrix} + \frac{\sqrt{3}}{2} \begin{bmatrix} 0 & 1 \end{bmatrix} \right) &= 0 \\
    &\Leftrightarrow \left( \alpha_1 \frac{\sqrt{3}}{2} + \alpha_2 \frac{i}{2} \right) \begin{bmatrix} 1 & 0 \end{bmatrix} + \left(\alpha_1 \frac{i}{2} + \alpha_2 \frac{\sqrt{3}}{2}\right) \begin{bmatrix} 0 & 1 \end{bmatrix} &= 0
\end{align*}

vì tập hợp $\left\{ \begin{bmatrix} 1 & 0 \end{bmatrix},  \begin{bmatrix} 0 & 1 \end{bmatrix}\right\}$ là tập cơ sở nên biểu thức trên chỉ xảy ra khi
\[
\begin{cases}
    \alpha_1 \frac{\sqrt{3}}{2} + \alpha_2 \frac{i}{2} = 0 \\
    \alpha_1 \frac{i}{2} + \alpha_2 \frac{\sqrt{3}}{2} = 0
\end{cases} \Leftrightarrow
\begin{cases}
    \alpha_1 = 0 \\
    \alpha_2 = 0
\end{cases}
\]
$\Rightarrow$ $\alpha_1 a + \alpha_2 b = 0$ khi và chỉ khi $\alpha_1 = \alpha_2 = 0 \Leftrightarrow B = \left\{a, b\right\}$ độc lập tuyến tính. 

Ta có:
\[
\begin{cases}
    \ket{a} = \frac{\sqrt{3}}{2} 
    \begin{bmatrix}
        1 \\
        0
    \end{bmatrix} + 
    \frac{i}{2}
    \begin{bmatrix}
        0 \\
        1
    \end{bmatrix} = 
    \begin{bmatrix}
        \frac{\sqrt{3}}{2} \\
        \frac{i}{2}
    \end{bmatrix} \\
    \ket{b} = \frac{i}{2} 
    \begin{bmatrix}
        1 \\
        0
    \end{bmatrix} + 
    \frac{\sqrt{3}}{2}
    \begin{bmatrix}
        0 \\
        1
    \end{bmatrix} = 
    \begin{bmatrix}
        \frac{i}{2} \\
        \frac{\sqrt{3}}{2}
    \end{bmatrix}
\end{cases}
\]
\[
\Rightarrow
\begin{cases}
    \braket{a}{b} = \ket{a}^\dagger \ket{b} = 
    \begin{bmatrix}
        \frac{\sqrt{3}}{2} & -\frac{i}{2}
    \end{bmatrix}
    \begin{bmatrix}
        \frac{i}{2} \\
        \frac{\sqrt{3}}{2}
    \end{bmatrix} = 0 \\
    \|a\| = \sqrt{\left(\frac{\sqrt{3}}{2}\right)^2 + \left(\frac{1}{2}\right)^2} = 1 \\
    \|b\| = \sqrt{\left(\frac{1}{2}\right)^2 + \left(\frac{\sqrt{3}}{2}\right)^2} = 1
\end{cases}
\]
$\Rightarrow B = \left\{a, b\right\}$ là cơ sơ trực chuẩn trong không gian $\mathbb{C}^2$.

$\Rightarrow$ toạ độ của $\ket{\phi}, \ket{\psi}$ trong cơ sở trực chuẩn $B$ là:
\[
\begin{cases}
    \left[\ket{\phi}\right]_{B} = \left(\braket{a}{\phi}, \braket{b}{\phi}\right) = \left(\frac{3}{4} - \frac{i}{4}, \frac{\sqrt{3}}{4}i - \frac{\sqrt{3}}{4}\right) \\
    \left[\ket{\psi}\right]_{B} = \left(\braket{a}{\psi}, \braket{b}{\psi}\right) = \left(\frac{\sqrt{3}}{3} + \frac{1}{6} - \frac{i}{3}, \frac{i}{3} + \frac{\sqrt{3}}{6} + \frac{i\sqrt{3}}{3}\right)
\end{cases}
\]