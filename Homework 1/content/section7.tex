\section{Bài 7}

\subsection{Đề bài}

Kiểm tra các vector sau có phân tách được
\begin{itemize}
    \item[(a)] $\ket{\phi_1} = \frac{1}{2} \left(\ket{00} - \ket{01} + \ket{10} - \ket{11}\right)$.
    \item[(b)] $\ket{\phi_2} = \frac{1}{2\sqrt{2}} \left(\sqrt{3}\ket{00} - \sqrt{3}\ket{01} + \ket{10} - \ket{11}\right)$.
    \item[(c)] $\ket{\phi_3} = \frac{1}{\sqrt{2}}\left(\ket{10} + i\ket{11}\right)$.
    \item[(d)] $\ket{\phi_4} = \frac{1}{\sqrt{3}}\ket{0+} + \sqrt{\frac{2}{3}}\ket{1-}$.  
\end{itemize}
\subsection{Lời giải}

\subsubsection{Phần (a)}
Ta có:
\begin{align*}
    \ket{\phi_1} &= \frac{1}{2} \begin{bmatrix} 1 \\ -1 \\ 1 \\ -1 \end{bmatrix} \\
    &= \frac{1}{2} \begin{bmatrix} 1 \\ 1 \end{bmatrix} \otimes \begin{bmatrix} 1 \\ -1 \end{bmatrix} \\
    &= \begin{bmatrix} \frac{1}{\sqrt{2}} \\ \frac{1}{\sqrt{2}} \end{bmatrix} \otimes \begin{bmatrix} \frac{1}{\sqrt{2}} \\ -\frac{1}{\sqrt{2}} \end{bmatrix} \\
    \Rightarrow \ket{\phi_1} &= \ket{+} \otimes \ket{-}
\end{align*}

Vậy vector $\ket{\phi_1}$ phân tách được.

\subsubsection{Phần (b)}
Ta có:
\begin{align*}
    \ket{\phi_2} &= \frac{1}{2\sqrt{2}} \begin{bmatrix} \sqrt{3} \\ -\sqrt{3} \\ 1 \\ -1 \end{bmatrix} \\
    &= \frac{1}{2\sqrt{2}} \begin{bmatrix} \sqrt{3} \\ 1 \end{bmatrix} \otimes \begin{bmatrix} 1 \\ -1 \end{bmatrix} \\
    &= \begin{bmatrix} \frac{\sqrt{3}}{2} \\ \frac{1}{2} \end{bmatrix} \otimes \begin{bmatrix} \frac{1}{\sqrt{2}} \\ -\frac{1}{\sqrt{2}} \end{bmatrix} \\
    \Rightarrow \ket{\phi_2} &= \left(\frac{\sqrt{3}\ket{0} + \ket{1}}{2}\right) \otimes \ket{-}
\end{align*}

Vậy vector $\ket{\phi_2}$ phân tách được.

\subsubsection{Phần (c)}
Ta có:
\begin{align*}
    \ket{\phi_3} &= \frac{1}{\sqrt{2}} \begin{bmatrix} 0 \\ 0 \\ 1 \\ i \end{bmatrix} \\
    &= \frac{1}{\sqrt{2}} \begin{bmatrix} 0 \\ 1 \end{bmatrix} \otimes \begin{bmatrix} 1 \\ i \end{bmatrix} \\
    &= \begin{bmatrix} 0 \\ 1 \end{bmatrix} \otimes \begin{bmatrix} \frac{1}{\sqrt{2}} \\ \frac{i}{\sqrt{2}} \end{bmatrix} \\
    \Rightarrow \ket{\phi_3} &= \ket{1} \otimes \ket{i}
\end{align*}

Vậy vector $\ket{\phi_3}$ phân tách được.

\subsubsection{Phần (d)}

Giả sử $\ket{\phi_4}$ tách được, ta có:
\[
\ket{\tau} = \begin{bmatrix} a \\ b \end{bmatrix}, \ket{\psi} = \begin{bmatrix} c \\ d \end{bmatrix} \in \mathbb{C}^2
\]
sao cho
\[
\ket{\phi_4} = \ket{\tau}\ket{\psi} \Leftrightarrow \begin{bmatrix} \frac{1}{\sqrt{6}} \\ \frac{1}{\sqrt{6}} \\ \frac{1}{\sqrt{3}} \\ \frac{-1}{\sqrt{3}}\end{bmatrix} = \begin{bmatrix} a \\ b \end{bmatrix} \otimes \begin{bmatrix} c \\ d \end{bmatrix} = \begin{bmatrix} ac \\ ad \\ bc \\ bd \end{bmatrix} \Rightarrow 
\begin{cases}
    ac = ad = \frac{1}{\sqrt{6}} \\
    bc = -bd = \frac{1}{\sqrt{3}}
\end{cases} (a, b, c, d \neq 0)
\]
\[
\Rightarrow \begin{cases}
c = d = \frac{1}{a\sqrt{6}} \\
c = -d = \frac{1}{b\sqrt{3}}
\end{cases} \Rightarrow
(c = d) \vee (c = -d) \Leftrightarrow d = -d = 0
\]
Vô lý vì $a, b, c, d \neq 0$, suy ra $\ket{\phi_4}$ không tách được.